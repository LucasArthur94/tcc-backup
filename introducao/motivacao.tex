\chapter{Motivação}

\section{Contexto}
Atualmente, o gerenciamento das duas disciplinas é realizado pelo Tidia-Ae, onde apenas os professores coordenadores da disciplina podem acessar o andamento da matéria, sem a participação dos demais envolvidos, em especial dos orientadores. Além disso, para compensar a ausência da participação no Tidia-Ae, os coordenadores precisam envolver os orientadores de maneira externa, para acompanhar o real \textit{status} do projeto, o que não é uma solução otimizada. Por fim, todo o processo de montagem de grupos, construção de bancas, parcerias para o evento e avaliações teóricas e práticas é realizada de maneira manual, o que gera um trabalho extenso para os coordenadores da disciplina.


\section{Problemas}
Além disso, não há um acompanhamento do status do projeto pelo Tidia-Ae, o que torna ele um simples repositório. O orientador acompanha o andamento do projeto apenas por intermédio do aluno, o que nem sempre gera uma abordagem eficiente, dado que ele depende do retorno do próprio aluno para saber atualizações, além de não ter uma central de fácil acesso para o orientador analisar os documentos relacionados.

Por fim, não há um sistema unificado que facilite os alunos que cursam a disciplina de consultar monografias anteriores de maneira estruturada e completamente on-line. O sistema visa, inicialmente, atacar essas demandas, de maneira a melhorar o andamento da disciplina como um todo. Futuramente, pode ser utilizada para outras disciplinas com estruturas semelhantes ao projeto de formatura.
