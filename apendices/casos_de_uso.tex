\chapter{Casos de Uso}\label{chap:use-case-appendix}

\section{Introdução}

Neste apêndice consta os casos de uso escritos para o sistema em questão, usando o padrão explicado no capítulo de casos de uso\cite{funpar2001}.

\subsection{Cadastrar disciplinas}


\begin{enumerate}
    \item Breve Descrição


Coordenadores cadastram a disciplina, os alunos participantes e as atividades.


    \item Fluxo de Eventos

\begin{enumerate}
    \item Fluxo Básico

\begin{itemize}
    \item Coordenador insere disciplina, com os seguintes dados:

\begin{itemize}
    \item Modalidade (Sem/Quad)

    \item Data de início e data de término


\end{itemize}
    \item Coordenador importa planilha com alunos participantes, com os seguintes dados dos alunos

\begin{itemize}
    \item Nome

    \item E-mail

    \item Nº USP


\end{itemize}
    \item Coordenador insere uma nova atividade da disciplina, com os seguintes dados

\begin{itemize}
    \item Nome

    \item Data de entrega e arquivos relacionados

    \item Peso da atividade


\end{itemize}
    \item Sistema salva alunos novos, dispara e-mail ao aluno com seu acesso (login e senha) e vincula os existentes à disciplina e salva a disciplina
\end{itemize}

    \item Fluxos Alternativos

\begin{enumerate}
    \item Data de início é posterior a de término

\begin{enumerate}
    \item Sistema exibe novamente tela de cadastro da disciplina, alertando sobre erro


\end{enumerate}
    \item E-mail é inválido

\begin{enumerate}
    \item Sistema exibe novamente tela de cadastro da disciplina, alertando sobre erro


\end{enumerate}
    \item E-mail retornou

\begin{enumerate}
    \item Sistema envia e-mail ao coordenador com o aluno problemático




\end{enumerate}
\end{enumerate}
\end{enumerate}
    \item Requisitos Especiais


Não há


    \item Pré-condição


Coordenador deve estar logado


    \item Pós-condição

    Não há
\end{enumerate}

\subsection{Editar disciplinas}


\begin{enumerate}
    \item Breve Descrição


Coordenadores editam disciplinas, cadastrando atividades, editando alunos etc.


    \item Fluxo de Eventos

\begin{enumerate}
    \item Fluxo Básico

\begin{itemize}
    \item Coordenador edita início e término da disciplina, alunos participantes, com novos alunos ou removendo os já participantes

    \item Coordenador insere uma nova atividade da disciplina, com os seguintes dados

\begin{itemize}
    \item Nome

    \item Data de entrega

    \item Arquivos relacionados

    \item Peso da atividade


\end{itemize}
    \item Sistema salva novas atividades e as mudanças da disciplina
\end{itemize}

    \item Fluxos Alternativos



\end{enumerate}
    \item Requisitos Especiais


Não há


    \item Pré-condição


Coordenador deve estar logado


    \item Pós-condição

    Não há
\end{enumerate}

\subsection{Cadastrar professores}


\begin{enumerate}
    \item Breve Descrição


Coordenadores cadastram professores do departamento que podem orientar/co-orientar.


    \item Fluxo de Eventos

\begin{enumerate}
    \item Fluxo Básico

\begin{itemize}
    \item Coordenador insere os dados do professor

\begin{itemize}
    \item Nome 

    \item Número USP

    \item E-mail


\end{itemize}
    \item Sistema salva o professor, disparando e-mail para o professor cadastrado
\end{itemize}


    \item Fluxos Alternativos



\end{enumerate}
    \item Requisitos Especiais


Não há


    \item Pré-condição


Coordenador deve estar logado


    \item Pós-condição

    Não há
\end{enumerate}




















\subsection{Cadastrar grupos de trabalhos}


\begin{enumerate}
    \item Breve Descrição


Coordenadores cadastram os grupos com os temas e os orientadores, com a confirmação da participação do orientador no grupo.


    \item Fluxo de Eventos

\begin{enumerate}
    \item Fluxo Básico

\begin{itemize}
    \item Coordenador insere os dados do grupo

\begin{itemize}
    \item Título

    \item Alunos

    \item Orientador

    \item Co-orientador


\end{itemize}
    \item Sistema salva grupo e envia e-mail para o orientador, co-orientador e alunos

    \item Orientador valida participação no grupo

    \item Co-orientador valida participação no grupo
\end{itemize}


    \item Fluxos Alternativos

\begin{enumerate}
    \item Grupo não tem orientador

\begin{enumerate}
    \item Sistema cadastra grupo, enviando e-mail para os alunos com o aviso de urgência na escolha do orientador


\end{enumerate}
    \item Grupo não tem aluno

\begin{enumerate}
    \item Sistema retorna para a tela de cadastro de grupo, avisando sobre o erro de ausência de alunos
\end{enumerate}
\end{enumerate}


    \item Requisitos Especiais


Não há


    \item Pré-condição


Coordenador deve estar logado, alunos, orientadores e co-orientadores cadastrados


    \item Pós-condição

    Grupo confirmado
\end{enumerate}
\end{enumerate}

\subsection{Login}


\begin{enumerate}
    \item Breve Descrição


Alunos, Orientadores, Co-orientadores e Coordenadores acessam sistema de maneira tradicional ou via Senha Única USP (para pertencentes à USP).


    \item Fluxo de Eventos

\begin{enumerate}
    \item Fluxo Básico

\begin{itemize}
    \item Sistema mostra campos de login e senha

    \item Usuário insere seu e-mail e sua senha

    \item Sistema valida e-mail e senha

    \item Usuário acessa sistema
\end{itemize}


    \item Fluxos Alternativos

\begin{enumerate}
    \item Usuário erra credenciais

\begin{enumerate}
    \item Sistema retorna para tela de acesso ao sistema, exibindo mensagem de erro

    \item Retorna normalmente à situação de mostrar campos de login e senha


\end{enumerate}
    \item Usuário realiza login pela Senha Única da USP

\begin{enumerate}
    \item Usuário seleciona $``$Acessar pela Senha Única USP$"$ 

    \item Usuário é redirecionado para os sistemas USP

    \item Retorna para a situação de acesso ao sistema
\end{enumerate}
\end{enumerate}


    \item Requisitos Especiais


Integração via Shibboleth com os Sistemas USP


    \item Pré-condição


Não há


    \item Pós-condição

    Usuário dentro do sistema
\end{enumerate}
\end{enumerate}


\subsection{Entregar atividade}


\begin{enumerate}
    \item Breve Descrição


Alunos submetem no Google Drive arquivos da atividade para a leitura do orientador, co-orientador e coordenadores. Orientador e co-orientador revisam e dão seu aval de aprovação com a documentação.


    \item Fluxo de Eventos

\begin{enumerate}
    \item Fluxo Básico

\begin{itemize}
    \item Aluno submete arquivos nos respectivos espaços de atividades, que são carregados no Google Drive e deixa comentários adicionais sobre a entrega

    \item Sistema salva a entrega com o status da entrega da atividade para $``$Não avaliada$"$ 

    \item Sistema envia e-mail para Orientador e Co-orientador avisando de submissão

    \item Orientador baixa documentos submetidos

    \item Orientador avalia a entrega, faz comentários aos alunos e comentários exclusivos à coordenação com notas

    \item Sistema salva entrega e dispara e-mail com o resultado da avaliação para os alunos
\end{itemize}


    \item Fluxos Alternativos

\begin{enumerate}
    \item Data de submissão expirou (1)

\begin{enumerate}
    \item Aluno não consegue interagir com atividade, encerrando fluxo


\end{enumerate}
    \item Co-orientador realiza fluxo de revisão, antes do Orientador (4)

\begin{enumerate}
    \item Passos 5 - 7 ocorrem normalmente, trocando Orientador por Co-orientador


\end{enumerate}
    \item Co-orientador realiza fluxo de revisão, após Orientador (8)

\begin{enumerate}
    \item Sistema exibe detalhes da entrega, porém não permite edições do lado do Co-orientador, encerrando fluxo


\end{enumerate}
    \item Aluno submete nova entrega da atividade após ter feito uma submissão (1)

\begin{enumerate}
    \item Aluno vê status da avaliação

    \item Aluno realiza nova entrega da atividade, repetindo o caso de uso


\end{enumerate}
    \item Aluno submete entrega da atividade quando alguém do grupo já submeteu (1)

\begin{enumerate}
    \item Sistema exibe detalhes da entrega já realizada

    \item Aluno pode realizar nova entrega da atividade, passando por cima da entrega anterior e repetindo o caso de uso
\end{enumerate}
\end{enumerate}

    \item Requisitos Especiais


Não há


    \item Pré-condição


Atores devem estar autenticados e atividade deve estar cadastrada no sistema


    \item Pós-condição
    
    Não há
\end{enumerate}
\end{enumerate}


\subsection{Construir bancas práticas}


\begin{enumerate}
    \item Breve Descrição


Coordenadores selecionam os participantes da banca prática, já cadastrados no sistema, e os notifica com comentários sobre o evento.


    \item Fluxo de Eventos

\begin{enumerate}
    \item Fluxo Básico

\begin{itemize}
    \item Coordenador informa o dia da feira, a disciplina correspondente e as salas disponíveis para o evento, além de determinar o peso da avaliação da banca

    \item Sistema salva evento

    \item Coordenador seleciona evento recém criado.

    \item Sistema lista grupos do evento.

    \item Coordenador seleciona o grupo que deseja atribuir a sala e os convidados.

    \item Coordenador seleciona os convidados que avaliarão o grupo

    \item Sistema salva banca e envia e-mail para os convidados externos, avisando-os sobre sua participação

    \item Convidado acessa sistema e confirma participação na banca
\end{itemize}

    \item Fluxos Alternativos

Não há

\end{enumerate}
    \item Requisitos Especiais


Não há


    \item Pré-condição


Atores devem estar autenticados e grupo de trabalho deve estar cadastrado no sistema


    \item Pós-condição

    Não há
\end{enumerate}












\subsection{Construir bancas teóricas}


\begin{enumerate}
    \item Breve Descrição


Coordenadores escolhem participantes da banca teórica do grupo, selecionam o presidente da banca, realizam o agendamento do horário, validando inconsistências (participante já possui horário ocupado) e notificam os participantes por e-mail.


    \item Fluxo de Eventos

\begin{enumerate}
    \item Fluxo Básico

\begin{itemize}
    \item Coordenador informa o dia da banca, a disciplina correspondente e as salas disponíveis para o evento

    \item Sistema salva evento

    \item Coordenador seleciona evento recém criado.

    \item Sistema lista grupos do evento.

    \item Coordenador seleciona o grupo que deseja atribuir a sala e os convidados.

    \item Coordenador seleciona os avaliadores da banca e o horário da avaliação. O orientador é um dos pré-selecionados por padrão
\end{itemize}

    \item Fluxos Alternativos

\begin{enumerate}
    \item Convidado externo já possui banca nesse dia e horário

\begin{enumerate}
    \item Sistema barra criação de banca, alertando sobre qual convidado já possui agenda ocupada


\end{enumerate}
    \item Sala está ocupada no horário selecionado

\begin{enumerate}
    \item Sistema barra criação de banca, alertando sobre qual sala já possui agenda ocupada
\end{enumerate}
\end{enumerate}

 \item Requisitos Especiais


Não há


    \item Pré-condição


Atores devem estar autenticados e grupo de trabalho deve estar cadastrado no sistema


    \item Pós-condição

    Não há
\end{enumerate}
\end{enumerate}


\subsection{Listar entregas}


\begin{enumerate}
    \item Breve Descrição


Técnicos recebem os arquivos de impressão, com normalização do título, separados por grupo.


    \item Fluxo de Eventos

\begin{enumerate}
    \item Fluxo Básico

\begin{itemize}
    \item Coordenador lista todas as entregas finais de impressão (\textit{banner} e \textit{press-release})

    \item Sistema salva as listas de arquivos finais, com o nome normalizado e envia por e-mail para o técnico responsável
\end{itemize}

    \item Fluxos Alternativos

\begin{enumerate}
    \item Grupo de trabalho está com arquivo faltante

\begin{enumerate}
    \item Sistema envia e-mail para o grupo com entrega faltante, avisando para regularizar a situação com urgência.
\end{enumerate}
\end{enumerate}

    \item Requisitos Especiais


Não há


    \item Pré-condição


Atores devem estar autenticados e grupo de trabalho deve estar cadastrado no sistema


    \item Pós-condição

    Não há
\end{enumerate}
\end{enumerate}

\subsection{Listar necessidades adicionais}


\begin{enumerate}
    \item Breve Descrição


Técnicos recebem necessidades adicionais revisadas pelos orientadores, separadas por grupo.


    \item Fluxo de Eventos

\begin{enumerate}
    \item Fluxo Básico

\begin{itemize}
    \item Coordenador lista todos os comentários das entregas finais

    \item Sistema salva a lista de comentários e a envia por e-mail para o técnico responsável
\end{itemize}

    \item Fluxos Alternativos



\end{enumerate}
    \item Requisitos Especiais


Não há


    \item Pré-condição


Atores devem estar autenticados e grupo de trabalho deve estar cadastrado no sistema


    \item Pós-condição

    Não há
\end{enumerate}






















\subsection{Avaliar projetos práticos}


\begin{enumerate}
    \item Breve Descrição


Participantes da banca prática avaliam os projetos que estão envolvidos, limitando avaliações até o final do dia.


    \item Fluxo de Eventos

\begin{enumerate}
    \item Fluxo Básico

\begin{itemize}
    \item Convidado externo acessa espaço da banca, com detalhes do grupo e as entregas finais

    \item Convidado preenche os comentários e as notas de acordo com cada critério estabelecido para avaliação de bancas práticas

    \item Convidado salva avaliação
\end{itemize}

    \item Fluxos Alternativos

\begin{enumerate}
    \item Convidado tenta submeter avaliação em dia diferente ao da banca prática

\begin{enumerate}
    \item Avaliação é barrada, avisando o convidado de que a avaliação só pode ser feita exclusivamente no dia



\end{enumerate}
\end{enumerate}
\end{enumerate}
    \item Requisitos Especiais


Não há


    \item Pré-condição


Atores devem estar autenticados e grupo de trabalho deve estar cadastrado no sistema


    \item Pós-condição

    Não há
\end{enumerate}
















\subsection{Avaliar monografias teóricas}


\begin{enumerate}
    \item Breve Descrição


Participantes da banca teórica avaliam as monografias que estão envolvidas, gerando comentários e definindo o status do trabalho (aprovado, aprovado com correções, recuperação e reprovado), limitando avaliações até o final do dia.


    \item Fluxo de Eventos

\begin{enumerate}
    \item Fluxo Básico

\begin{itemize}
    \item Participante da banca acessa espaço da banca, com detalhes do grupo e as entregas parciais e finais.

    \item Participante preenche os comentários e as notas de acordo com cada critério estabelecido para avaliação de bancas teóricas

    \item Participante salva avaliação, com seu parecer para aprovação da monografia

    \item Presidente da banca vê avaliações dos outros envolvidos e determina o parecer final para a banca, com comentários para o grupo
\end{itemize}

    \item Fluxos Alternativos

Não há



\end{enumerate}
    \item Requisitos Especiais


Não há


    \item Pré-condição


Atores devem estar autenticados e grupo de trabalho deve estar cadastrado no sistema


    \item Pós-condição

    Não há
\end{enumerate}


\subsection{Calcular nota final dos projetos}


\begin{enumerate}
    \item Breve Descrição


Coordenadores determinam a fórmula para calcular as notas finais, com base nas entregas parciais durante as duas disciplinas e o sistema calcula as notas finais de todos os grupos participantes.


    \item Fluxo de Eventos

\begin{enumerate}
    \item Fluxo Básico

\begin{itemize}
    \item Coordenador escolhe as duas disciplinas (TCC1 e TCC2), a banca teórica e a feira prática que deseja obter as entregas

    \item Sistema salva fórmula de avaliação

    \item Sistema lista todos os grupos, com as notas calculadas e os estados de avaliação de cada grupo
\end{itemize}

    \item Fluxos Alternativos

\begin{enumerate}
    \item Coordenador não preenche algum dos campos necessários

\begin{enumerate}
    \item Sistema barra criação de fórmula de avaliação, avisando os campos faltantes


\end{enumerate}
    \item Grupo tem alguma avaliação faltante

\begin{enumerate}
    \item Sistema exibe grupo, mas com campo de Não Avaliado


\end{enumerate}
    \item Grupo vai para recuperação na avaliação da banca teórica

\begin{enumerate}
    \item A nota calculada é a nota da banca teórica, passando todas as outras avaliações realizadas ao longo das disciplinas


\end{enumerate}
    \item Grupo é reprovado na avaliação da banca teórica

\begin{enumerate}
    \item A nota calculada é a nota da banca teórica, passando todas as outras avaliações realizadas ao longo das disciplinas

Não há



\end{enumerate}
\end{enumerate}
\end{enumerate}
    \item Requisitos Especiais


Não há


    \item Pré-condição


Atores devem estar autenticados e grupo de trabalho deve estar cadastrado no sistema


    \item Pós-condição

    Não há
\end{enumerate}